\documentclass[12pt]{report}

\usepackage{listings}
\usepackage{xcolor}
\usepackage{mathtools}

\usepackage[a4paper, total={7.25in, 10.5in}]{geometry}
\renewcommand{\arraystretch}{1.5}
\newcommand\tab[1][1cm]{\hspace*{#1}}
% \newcommand\indent[1][1cm]{\hspace*{#1}}
%New colors defined below
\definecolor{codegreen}{rgb}{0,0.6,0}
\definecolor{codegray}{rgb}{0.5,0.5,0.5}
\definecolor{codepurple}{rgb}{0.58,0,0.82}
\definecolor{backcolour}{rgb}{0.95,0.95,0.92}

\lstdefinestyle{mystyle}{
  backgroundcolor=\color{backcolour},
  commentstyle=\color{codegreen},
  keywordstyle=\color{magenta},
  numberstyle=\tiny\color{codegray},
  stringstyle=\color{codepurple},
  basicstyle=\ttfamily,
  breakatwhitespace=false,
  breaklines=true,
  captionpos=b,
  keepspaces=true,
  numbers=left,
  numbersep=5pt,
  showspaces=false,
  language=Python,
  showstringspaces=false,
  showtabs=false,
  tabsize=2,
}

\usepackage{array}
\newcolumntype{A}{>{\centering\arraybackslash}m{3cm}}
\newcolumntype{L}{>{\centering\arraybackslash}m{4cm}}
\newcolumntype{M}{>{\centering\arraybackslash}m{5cm}}
\newcolumntype{R}{>{\centering\arraybackslash}m{7cm}}
\lstset{style=mystyle}


% \title{Assignment 1: Basics}
% \author{Language: Python}
% \date{Due: January 22, 2020}

\begin{document}
\begin{titlepage}
    \begin{center}
\vspace*{9cm}
  \begin{LARGE}
        Assignment 1: Basics    
    \end{LARGE}

    \vspace*{0.5cm}
    
    \begin{Large}
        Language: Python
    \end{Large}

    \vspace*{0.5cm}
\end{center}
\end{titlepage}
\section*{Topics and Keywords / Symbols Used}

This assignment covers the topics in the table. 
There are certain keywords that the given topic uses which is listed below the given topic. 
Not all the symbols and keywords will be used, but many will.


\subsection*{Table}

\begin{tabular}{ | c | c | c | c | c | c | c |}
    \hline
    Variables & Collections & Conditional Statements & Iteration & Functions & Exceptions & Modules \\
    \hline
    \noalign{\smallskip}\noalign{\smallskip}\noalign{\smallskip}\noalign{\smallskip}
    \hline
    = & [] & if(c): & for & def f(p):& try: & import \\
    \hline
    input(str) & \{\} & elif(c): & while(c) & return & except exc: & from \\
    \hline
    int(e) & pop(i?) & else: & range(o,n,${\Delta}$) &&& \\
    \hline
    & append(e) && in &&&\\ 
    \hline
    & len(l) &&&&& \\
    \hline

\end{tabular}

\subsection*{Legend}

\begin{tabular}{| c | c |}
    \hline
    Character & Definition \\
    \hline
    c & some condition must go here \\
    \hline
    e & some element or item must go here \\
    \hline
    o & some original value \\
    \hline
    n & some ending value \\
    \hline
    ${\Delta}$ & some changing factor \\ 
    \hline
    exc & a specific exception \\
    \hline
    f & function name \\
    \hline
    p & parameter \\
    \hline
    str & some string \\
    \hline
    i & index \\
    \hline
    l & list \\
    \hline
    ${?}$ & optional value \\
    \hline
\end{tabular}

\subsection*{Note on Topics}
Files aren't included in any of these parts as that goes beyond the pure basics. 
This assignment is an assignment to apply your knowledge and understanding of the basics of Python. 
The next assignment will include harder or more complicated topics such as Files and Recursion.
For now, as this is to build your foundation, the topics will be limited to more foundational topics and aspects of Python.
The topics and keywords are generally limited to the list above. 
The aforementioned list isn't an extensive list of all the keywords you may or will use, but they contain the general terms you will likely require.

\newpage

\section*{Explanation}

\hspace*{0.5cm} This assignment has a couple parts. 
Read below for the descriptions of the parts and their breakdowns. 
The parts aren't related so you don't have to do them in any particular order, but I'd advice going down the parts. 
Not all parts use everything you've learned, but most of the concepts in the list above are included.

At the very bottom, there's a reference guide attached. This reference guide includes a brief writeup of what the keywords or functions do along with what they return, if they return anything.

\section*{Task}

The tasks are listed below. The descriptions are on the subsequent pages.

\tab Part 1 : Mathematical Functions

\tab Part 2 : Shopping List

\tab Part 3 : Phone Contacts


\section*{Online Resources}
\hspace*{0.55cm}I recognize that I can't prevent you from searching up everything and using StackOverflow or similar resources.
I do advise using these resources, but I also recognize that you hardly learn when you use them. 
If you're going to use any online resource, use a comment and say where you got it from. 
I'm most likely going to still ask you to code it differently and possibly code it again, but only if I suspect that you don't know what the code you copied does.

Make an attempt to code it all on your own. 
Review the notes from our sessions as I believe they cover 
the important aspects for these projects. 

\section*{Questions}
If you have any questions, feel free to contact me.
I'll be checking my email and my socials over the break, 
so you should be able to contact me with anything we have, 
except by text messages as I'll be back on my HK number.

\section*{Submission}
When you're done with everything, or if you just want me to check through any part, send me the compressed version of all the files (or the parts) you have. 
Tell me which parts you've done so I know what I'm looking at.
I'll run the code and see how you did everything. I'll send you feedback regarding your implementation, naming conventions, or potential problems that may have been missed.

\section*{Good Luck!}
Now that that's all out of the way, good luck with everything! 
If you can solve all of this by the time we get back from the break, you're going to be perfectly fine going into CS110. 
If you're struggling, let me know and honestly ask me questions. 
I'm happy to answer any questions you may have.

\newpage

\subsection*{Part 1: Mathematical Functions}

In this part, you will write a number of functions progressively getting more difficult.
These functions will be laid out in the following table with the name of the function, what parameters are used, what the return type and values are, and what exception may need to be caught. 
The only two builtin functions or variables you may use are ${math.sqrt}$ and ${math.pi}$. 

${}$

\begin{tabular}{| c | c | c | M | c |}
    \hline
    Function name & Parameters & Return Type & Return Value / Description & Exception\\
    \hline
    add & (a,b) & int & sum of a and b & N/A \\
    \hline
    subtract & (a,b) & int & difference between a and b & N/A \\
    \hline
    multiply & (a,b) & int & product of a and b & N/A \\
    \hline
    divide & (a,b) & int & quotient of a over b & ZeroDivisionError \\
    \hline
    pow & (a,b) & int & power of ${a^b}$ & N/A \\
    \hline
    pythagoras & (a,b) & int & the value of ${c}$ \newline ${c = \sqrt{a^2 + b^2}}$ & N/A \\
    \hline
    dist & (x1, y1, x2, y2) & int & distance between coordinates \newline $\sqrt{(x1 - x2)^2 + (y1 - y2)^2}$ & N/A \\
    \hline
    quad & (a,b,c) & [list] & possible values as a list for values of a, b, and c \newline $\frac{-b\pm\sqrt{b^2-4ac}}{2a}$& ZeroDivisionError \\
    \hline
    degtorad & (deg) & int & the radian equivalent to a degrees value \newline ${rad = \frac{deg * \pi}{180}}$ & N/A\\
    \hline
    isprime & (a) & boolean & True if number is prime, False otherwise & N/A \\
    \hline
    sum & (n) & int & sum of the values from 0 to n \newline (${\sum_{i = 0}^{n}{i}}$) & N/A \\
    \hline
    fact & (n) & int & factorial of n (${n!}$) & N/A \\
    \hline
    sumofsquares & (n) & int & sum of the squared values from 0 to n \newline (${\sum_{i = 0}^{n}{i^2}}$) & N/A\\
    \hline
\end{tabular}


\newpage
\subsection*{Part 2: Shopping List}

In this part, you are going to write a small program to serve as a shopping list. 
It will have an interactive component to it. You should have at least one variable of the list being used as the shopping list.
Shopping list should start as an empty shopping list. Functions should be used to manipulate the list. You can implement as many helper functions, but the functions below must be present and functional.

${}$

\begin{tabular}{| c | c | c | L | c |}
    \hline
    Function Name & Parameters & Return Type & Return Value / Description & Potential Problems \\
    \hline
    contains & (e) & boolean & True if found in list, False otherwise & N/A \\
    \hline
    add & (e) & boolean & Add element to list, return True if was added, False if it couldn't be added & N/A \\
    \hline
    remove & (e) & string & Item removed & Item Not In List \\
    \hline
    find & (e) & int & Index of item if found, -1 if not found & N/A \\
    \hline
    clear & N/a & N/a & Clears or resets the list & N/a \\
    \hline
    printall & N/a & N/a & Print out values in list as a visual list \newline Example: \newline Shopping List: \newline - Apples \newline - Tomato Sauce & List is empty \\
    \hline
    isover* & N/a & boolean & End interactive aspect & N/a \\
    \hline
    getinput* & N/a & str & Get the input from the user & N/a \\
    \hline
    validate* & str & int & Used to receive action from user as to what to do. If \lstinline[language=Python]!ValueError!,  Must find a way to validate what the user inputs and keep asking until they give a valid response & \lstinline[language=Python]!ValueError! \\
    \hline
\end{tabular}

${}$

Note: * denotes function is used for interactive aspect.

\newpage

\subsection*{Part 3: Phone Contacts}
In this part, you will create an empty dictionary that will store the contacts of a person. 
In essence, you are writing a system similar to how your phone or email keeps track of your contacts.

The dictionary named \lstinline[language=Python]{contacts} will store key:value pairs of: 

\lstinline[language=Python]{contactName}:\lstinline[language=Python]{[phoneNumber, address, email]}.

\begin{tabular}{| c | A | L | L | c |}
    \hline
    Func. Name & Params & Return Type & Description & Exception \\
    \hline
    contains & (name) & boolean & Checks if name is in dictionary, if found, return True, if not, return False & KeyError \\
    \hline
    add & (name, phoneNumber, address, email) & boolean & Adds to dictionary if not already in dictionary. return True if successful, False is record already found & N/a \\
    \hline
    remove & (name) & boolean & Removes record from dictionary, returns True if removed, False if not removable & KeyError \\
    \hline
    get & (name) & [list] & Returns records of the name if found, returns empty list if no record found & KeyError \\
    \hline
    printall & N/a & N/a & Print out all records in the dictionary with format of : \newline Name: \{name\}, Number: \{phoneNumber\}, Address: \{address\}, Email: \{email\} & N/a\\ 
    \hline
    sortkeys & (contacts) & [list] & Sort the keys of the contacts in alphabetical order & N/a \\
    \hline
    reordercontacts & (sortedkeys) & \{dictionary w/ sorted keys\} & Using the list of sorted keys, reorganize the contacts and return a new ordered contacts list & N/a \\
    \hline
\end{tabular}\\

${}$

Extra: Make it interactive using the interactive components / functions from Part 2!

\newpage

\section*{Reference Guide}
This page serves as a reference guide as to what each keyword or built in function does.\\
\begin{tabular}{| c | R | R |}
    \hline
    Keyword/Function & Purpose & Return / Crashed Exception \\
    \hline
    input & Takes in a string parameter to print to the user when requesting a value & Returns the value input by the user as a string \\
    \hline
    int & Takes in an element, attempts to cast to an integer & int value of element passed in, crashes with \lstinline[language=Python]!ValueError! if cannot cast element to an int \\
    \hline
    pop & Removes item from list based on index. Removes last if no parameter given. & element removed \\
    \hline
    append & Takes in an element and adds it into the list & Nothing is returned (void) \\
    \hline
    len & return length of list taken in & returns length of list \\
    \hline
    if & executes code if specific condition is met & N/a \\
    \hline
    elif & executes code if \lstinline[language=Python]!if! is not met, but given condition is met & N/a \\
    \hline
    else & executes code if \lstinline[language=Python]!if! and \lstinline[language=Python]!elif! clauses are not met, no condition is used & N/a \\
    \hline
    for & uses a variable provided to iterate through a range or collection & N/a \\
    \hline
    while & takes a condition and keeps looping until condition is met & N/a, can cause infinite loop \\
    \hline
    range & takes an original starting value, ending value, and some changing factor. Allows for iteration through a series of numbers & a sequence of iterable numbers \\
    \hline
    in & used for the \lstinline[language=Python]!for! loop to iterate through range or collection & N/a \\
    \hline
    def & defines a function with any number of parameters & N/a \\
    \hline
    return & allows a value to exit a function and be used elsewhere & N/a \\
    \hline
    try & attempts to execute code & N/a \\
    \hline
    except & if try fails, will go to except clause with specific exception that broke the code & N/a \\
    \hline
    import & imports a module & N/a \\
    \hline 
    from & used together with imports for shorter use \lstinline[language=Python]!from math import pi! allows for the use of \lstinline[language=Python]!pi! instead of \lstinline[language=Python]!math.pi! & N/a \\
    \hline
\end{tabular}

\end{document}